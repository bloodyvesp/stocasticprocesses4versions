\documentclass[12pt]{article}
\usepackage[utf8]{inputenc}
\usepackage{amsmath}
\usepackage{amsmath, amsfonts, amssymb}
\usepackage{amsmath,amssymb,amsfonts,latexsym,cancel}
\usepackage{amsmath}
\usepackage{amssymb}
\usepackage{amsthm}
\usepackage{appendix}
\usepackage{calrsfs}
\usepackage{latexsym}
\usepackage{rawfonts}
\title{Procesos estocásticos}
\date{}

\swapnumbers
\theoremstyle{plain}
\newtheorem*{teo}{Teorema}
\newtheorem{teorema}{\;\;\;Teorema}[section]
\newtheorem{lema}[teorema]{\;\;\;Lema}
\newtheorem{proposicion}[teorema]{\;\;\;Proposici\'on}

\theoremstyle{definition}
\newtheorem{definicion}[teorema]{\;\;\;Definici\'on}

\begin{document}
  \maketitle 

 \begin{definicion}
  \label{}
  Una \textbf{gr\'afica} $G$ es un par ordenado $(V,E)$ donde $V$ es un conjunto
  finito a cuyos elementos se les llama $v\acute ertices$ y $E$ es una familia de
  subconjuntos de $V$ con exactamente dos elementos a los que se les llama $aristas$. 
\end{definicion}
  % This is a comment; it is not shown in the final output.
  % The following shows a little of the typesetting power of LaTeX
  \begin{align}
    E &= mc^2                              \\
    m &= \frac{m_0}{\sqrt{1-\frac{v^2}{c^2}}}
  \end{align}
\end{document}
