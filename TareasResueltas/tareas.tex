\documentclass[a5paper,oneside]{amsart}
\usepackage[scale={.9,.9}]{geometry}
\usepackage{mathrsfs}
\theoremstyle{plain}
\newtheorem{theorem}{Theorem}
\newtheorem{lemma}{Lemma}
\newtheorem{corollary}{Corollary}
\newtheorem{proposition}{Proposition}
\newtheorem{conjecture}{Conjecture}
\theoremstyle{definition}
\newtheorem{problema}{Problema}
\newtheorem*{definition}{Definition}
\newtheorem*{remark}{Remark}
\title[Problemas de Procesos I]{Problemas de Procesos Estoc\'asticos I\\ Semestre 2013-II\\ Posgrado en Ciencias Matem\'aticas\\ Universidad Nacional Aut\'onoma de M\'exico}
\author{Ger\'onimo Uribe Bravo}
%\address{}
\usepackage[colorlinks,citecolor=blue,urlcolor=blue]{hyperref}
\input{definitions.tex}
%\usepackage[colorlinks,citecolor=blue,urlcolor=blue]{hyperref}
\begin{document}
\maketitle
\begin{problema}
Sean $\paren{X_n}_{n\in\na}$ un proceso estoc\'astico con valores reales y $A\subset\re$ un boreliano. Pruebe que si\begin{esn}
T_0=0\quad\text{y}\quad T_{n+1}=\min\set{n>T_n: X_n\in A}
\end{esn}entonces $T_n$ es un tiempo de paro para toda $n$ y $T_n\to \infty$ puntualmente conforme $n\to\infty$. 

\defin{Categor\'ias: } Tiempos de paro
\end{problema}
\begin{problema}[Lo que siempre tiene una posibilidad razonable de suceder lo har\'a; (casi seguramente)-- y pronto]
\emph{Tomado de \cite[E10.5, p.223]{MR1155402}}

Suponga que \(T\) es un tiempo de paro tal que para alg\'un \(N\in\na\) y \(\varepsilon>0\) se tiene que para toda \(n\in\na\):
 $$
 \p (T\leq N+ n|\F_n)>\varepsilon \text{ casi seguramente}
 $$
Al verificar la desomposici\'on
 $$
\p (T>kN)= \p (T>kN,T>(k-1)N),
 $$pruebe por inducci\'on que para cada \(k=1,2,\ldots\):
 $$
\p (T>kN)\leq \paren{1-\eps}^k. 
 $$Pruebe que \( \esp{T}<\infty \).
 
 \defin{Categor\'ias:} Tiempos de paro.
\end{problema}
\begin{problema}
\emph{Tomado de Mathematical Tripos, Part III, Paper 33, 2012, \url{http://www.maths.cam.ac.uk/postgrad/mathiii/pastpapers/}}

Sean $\paren{X_i,i\in\na}$ variables aleatorias independientes con $\proba{X_i=\pm 1}=1/2$. Sean $S_0=0$ y $S_n=\sum_{i=1}^n X_i$. 
\begin{enumerate}
\item Sea $T_1=\min\set{n\geq 0:S_n=1}$. Explique por qu\'e $T_1$ es un tiempo de paro y calcule su esperanza.
\item Mediante el inciso anterior, construya una martingala que converge casi seguramente pero no lo hace en $L_1$.
\item Sea $T=\min\set{n\geq 2:S_n=S_{n-1}}$ y $U=T-2$. ?`Son $T$ y $U$ tiempos de paro? Justifique su respuesta.
\item Para la variable $T$ que hemos definido, calcule $\esp{T}$. 
\end{enumerate}

\defin{Categor\'ias: } Tiempos de paro, problema de la ruina
\end{problema}

\begin{problema}[Extensiones del teorema de paro opcional]
Sea \(M=\paren{M_n,n\in\na}\) una (super)martingala respecto de una filtraci\'on \(\paren{\F_n,n\in\na}\) y sean \(S\) y \(T\) tiempos de paro.
\begin{enumerate}
                \item Pruebe que \(S\wedge T\), \(S+T\) y \(S\vee T\) son tiempos de paro.
                \item Sea \begin{esn}\F_T=\set{A\in\F:A\cap\set{T\leq n}\in\F_n\text{ para toda } n}\end{esn}es una \(\sigma\)-\'algebra, a la que nos referimos como la \(\sigma\)-\'algebra detenida en \(\tau\). Comente qu\'e puede fallar si \(T\) no es tiempo de paro. Pruebe que \(T\) es \(F_T\)-medible. 
                \item Pruebe que si \(T\) es finito, entonces \(M_T\) es \(\F_T\)-medible.
                \item Pruebe que si \(S\leq T\leq n\) entonces \(\F_S\subset\F_T\). Si adem\'as \(T\) es acotado entonces \(X_S,X_T\in L_1\) y \begin{esn}\espc{M_T}{\F_S}\leq M_S.\end{esn}
                \item Si \(X=\paren{X_n,n\in\na}\) es un proceso estoc\'astico \(\paren{\F_n}\)-adaptado y tal que \(X_n\in L_1\) y tal que para cualesquiera tiempos de paro acotados \(S\) y \(T\) se tiene que \(\esp{X_S}=\esp{X_T}\) entonces \(X\) es una martingala. Sugerencia: considere tiempos de paro de la forma \(n\indi{A}+(n+1)\indi{A^c}\) con \(A\in\F_n\).
\end{enumerate}

\defin{Categor\'ias: }Tiempos de paro, Muestreo opcional
\end{problema}
\bibliography{GenBib}
\bibliographystyle{amsalpha}
\end{document}